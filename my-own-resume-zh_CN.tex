% !TEX TS-program = xelatex
% !TEX encoding = UTF-8 Unicode
% !Mode:: "TeX:UTF-8"

\documentclass{resume}
\usepackage{zh_CN-Adobefonts_external} % Simplified Chinese Support using external fonts (./fonts/zh_CN-Adobe/)
%\usepackage{zh_CN-Adobefonts_internal} % Simplified Chinese Support using system fonts
\usepackage{linespacing_fix} % disable extra space before next section
\usepackage{cite}

\begin{document}
\pagenumbering{gobble} % suppress displaying page number

\name{陆健美 - 简历}

\basicInfo{
  \email{anysky130@163.com} \textperiodcentered\ 
  \phone{(+86) 186-681-13293} \textperiodcentered\ 
  \linkedin[lujianmei]{http://www.linkedin.com/in/lu-jianmei-55b686133/}}
 
\section{\faGraduationCap\  教育背景}
\datedsubsection{\textbf{湖北师范大学}, 湖北}{2005 -- 2009}
\textit{学士}\ 电子信息工程 

\section{\faUsers\ 工作经历}
\datedsubsection{\textbf{北京拓尔思信息技术股份有限公司} 杭州、青岛}{2009年3月 -- 2015年12月}
\role{软件开发工程师}{}
WEB后端开发
\begin{itemize}
\item 参与实际项目调研、设计、开发工作
\item 协助项目经理完成分配的开发任务
\item 项目小组任务管理
\end{itemize}

\datedsubsection{\textbf{北京拓尔思信息技术股份有限公司} 杭州、青岛}{2011年3月 -- 至今}
\role{项目经理}{}
\begin{itemize}
    \item 7年项目管理经验,参与各合作项目前期需求调研、投标方案材料、项目投标方案演讲;
    \item 项目落地技术方案规划、实施方案制定、技术原型设计;
    \item 项目进度管理,包含项目业务定期汇报、项目过程中资源及需求进度管理,及内部开发团队进度管理;
    \item 项目风险及质量成本管理,对项目的范围、技术难点、业务流程等过程产生的质量风险及成本控制;
    \item 项目内部团队及其它多团队协助管理,多团队管理过程中的协作流程、各团队风险质量进度统一监控管理,制定管理流程及监控监督各团队进度同步及实施过程质量风险控制;
    \item 客户需求、客户沟通及业务期望管理,确保客户的目标期望及项目的成功;
    \item 主要管理项目包含:某知名家电企业全球品牌网站建设及运营、售后电话中心知识库项目、家电企业积分商城/积分中心项目、某知名证券交易所官方网站及APP项目;
\end{itemize}

\datedsubsection{\textbf{北京拓尔思信息技术股份有限公司} 杭州、青岛}{2011年3月
  -- 2014年12月}
\role{海外项目经理}{}
\begin{itemize}
    \item 某知名家电企业全球网站项目管理大项目经理;
    \item 负责全球23个国家需求调研、项目管理规范、项目进度推进管理;
    \item 各国家建设推进、国内及海外产品、营销、第三方代理等多团队协作管理,推进各团队进度及质量;
    \item 各国家海外市场数字化品牌运营技术支持、平台互动应用功能规划及开发管理;
\end{itemize}

\datedsubsection{\textbf{北京拓尔思信息技术股份有限公司} 杭州、青岛}{2014年5月
  -- 至今}
\role{产品经理}{}
\begin{itemize}
    \item 3年产品经理经验,产品设计前期需求调研、梳理调研文档、及业务流程设计文档;
    \item 平台产品原型设计、相应配套流程及运营流程设计文档;
    \item 产品上线后运营阶段各业务运营反馈调研、流程优化方案;
    \item 协助技术实施方案规划及设计;
    \item 产品设计涉及企业服务知识库应用(生产、销售、售后、电话中心等)、某知名证券交易所官方APP、某知名媒体集团移动采编APP、知名家电企业商城前后台整体应用设计、知名家电企业多品牌积分商城平台、知名家电企业数据中心项目;
\end{itemize}

\datedsubsection{\textbf{北京拓尔思信息技术股份有限公司} 成都、青岛}{2016年2月
  -- 2016年10月}
\role{项目架构顾问}{}
\begin{itemize}
    \item 协助知名家电企业整体电商方案设计,包含后端产品中心、商品中心、活动中心、订单中心、结算中心业务及技术架构设计;
    \item 协助用户互动包含社区积分中心、积分商城设计架构解决方案;
    \item 协助搭建整合电话中心、售前售后知识库平台建设;
    \item 协助规划部运营团队建设,各平台相应安全、数据等统一解决方案;
    \item 协助搭建自动化DevOps技术平台;
\end{itemize}

\datedsubsection{\textbf{北京拓尔思信息技术股份有限公司} 青岛}{2015年10月
  -- 2016年6月}
\role{部门项目总监}{}
\begin{itemize}
    \item 部门内部运营管理,包含内部多团队间资源协调、内部运营工作流程优化;
    \item 项目集管理,项目主要进度推进、重点项目工作管理、及跨团队工作推进;
    \item 内部及外部客户周期性成果、运营汇报;
    \item 各项目资源输入及成果产出管理及分析;
    \item 工程师工作绩效精细化管理;
\end{itemize}


% Reference Test
%\datedsubsection{\textbf{Paper Title\cite{zaharia2012resilient}}}{May. 2015}
%An xxx optimized for xxx\cite{verma2015large}
%\begin{itemize}
%  \item main contribution
%\end{itemize}

\section{\faCogs\ 工作技能}
% increase linespacing [parsep=0.5ex]
1. 技术能力
\begin{itemize}[parsep=0.5ex]
    \item  精 通: LINUX
    \item  熟 悉: Java、C、Python、Database、各种常用Web应用开发中间件
    \item  精 通:  Emacs、Elisp
    \item  熟 悉: VIM
  \end{itemize}
    
2. 项目、项目集管理能力
\begin{itemize}[parsep=0.5ex]
    \item 7年以上专业项目管理经验
    \item 善于对项目、项目集多团队协助的大项目管理,善于处理协作间资源管理、进度风险控制
    \item 对项目进度、风险、成本、日常任务监控管理有丰富经验
    \item 关于使用各种项目管理工具协助监控及管理日常项目
  \end{itemize}
3. 产品设计能力 
\begin{itemize}[parsep=0.5ex]
    \item 3年以上产品设计能力,精通Axure/MockingBot/OmniGraffle/Sketch等原型设计工具
    \item 精于需求调研及整理、产品原型逻辑、业务流程优化设计,打破常规流程创新流程
    \item 有很好的产品业务架构能力,能针对不同场景业务设计相应业务产品
    \item 熟悉互联网产品的用户体验
  \end{itemize}

    
\section{\faInfo\ 其他}
% increase linespacing [parsep=0.5ex]
  \begin{itemize}[parsep=0.5ex]
    \item 英文沟通: 能熟练与海外沟通,熟练英文写作能力
    \item 学习: 快速学习能力,研究型学习者
    \item 抗压: 强抗压能力, 能适应海外工作
    \item 证书
      \item 2006年5月    SCJP
      \item 2010年5月 - 2012年4月 华尔街英语
      \item 2012年6月30  PMP
      \item 2016年7月 - 至今    CISSP(准备中)
    \item GitHub: http://github.com/lujianmei
    \end{itemize}

%% Reference
%\newpage
%\bibliographystyle{IEEETran}
%\bibliography{mycite}
\end{document}
